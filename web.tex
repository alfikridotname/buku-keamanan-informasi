\chapter{Keamanan Sistem Web}
World Wide Web (WWW, atau sering disingkat menjadi Web saja) merupakan aplikasi
yang dominan pada saat ini. Kepopulerannya setara dengan aplikasi email. Boleh
jadi kalau dilihat dari jumlah data yang dikirimkan melalui web atau melalui
email, maka jumlah data yang dikirim melalui web lebih besar.

Bahasan mengenai masalah keamanan sistem web sesungguhnya dapat menjadi sebuah
buku yang tersendiri. Bab ini akan membahas secara singkat masalah-masalah
tersebut.

\section{Sejarah Web}
WWW pada awalnya dikembangkan oleh Tim Berners-Lee ketika dia bekerja di CERN,
sebuah tempat penelitian terkait dengan Fisika di Swiss di akhir tahun 1980-an
(sekitar tahun 1989)~\footnote{Saya mengikuti perkembangan sejarah Web ini
karena secara tidak sengaja saya mengikutinya sejak awal. Ceritanya pada akhir
tahun 80-an tersebut saya bekerja di kampus University of Manitoba dan
``terpaksa'' menggunakan komputer NeXT. Komputer NeXT ini adalah produk yang
dikembangkan oleh Steve Jobs setelah dia ditendang dari Apple. Komputer NeXT
ini sangat langka karena termasuk mahal. Pada waktu itu tidak banyak yang mau
pakai karena softwarenya juga terbatas. Saya sedang mencari-cari software apa
saja yang tersedia dan siapa saja yang pakai ketika melihat Tim Berners-Lee
sedang mengembangkan WWW dengan menggunakan komputer NeXT-nya. Jadi kebetulan
saya termasuk yang ikut ngoprek WWW pada awal-awalnya.}.
CERN merupakan tempat berkumpulnya peneliti-peneliti yang berasal dari berbagai
tempat di seluruh dunia. Komputer dan software yang digunakan para peneliti
tersebut tentunya berbeda-beda. Demikian pula dokumen yang dihasilkannya
diletakkan pada komputer yang tersebar.

Pada saat itu protokol untuk distribusi informasi yang paling banyak digunakan
adalah FTP dan Gopher. Keduanya cukup baik untuk mengelola atau
mendistribusikan data (informasi) dalam sebuah direktori, misalnya. Namun orang
masih harus mencari dimana letak direktori tersebut (di komputer mana dan dalam
direktori mana). Proses ini membuat distribusi informasi tidak terlalu nyaman.

WWW terdiri dari dua komponen utama, yaitu standar format dokumen yang
disebut HTML (HyperText Markup Language) dan protokol HTTP (HyperText Transfer
Protocol). Keduanya dikembangkan oleh Tim Berners-Lee~\cite{webbook}.

Pada awalnya WWW tidak terlalu terkenal karena softwarenya terbatas. Pada suatu
saat ada mahasiswa kerja praktek di NCSA yang diminta untuk mengembangkan
browser untuk tiga platform; UNIX, Windows, dan Mac. Mahasiswa-mahasiswa
tersebut akhirnya mengembangkan browser yang diberi nama Mosaic. Setelah
tersedia di ketiga platform tersebut, maka WWW mulai banyak pengguna. Maka
mulai populerlah dia. Salah satu mahasiswa tersebut bernama Marc Andreessen,
yang kemudian akhirnya menjadi salah satu pengembang dari browser Netscape yang
nantinya juga menjadi cikal bakal dari Mozilla.

HTML merupakan sebuah standar yang dikembangkan dari SGML, Standard Generalized
Markup Language. Sebenarnya HTML merupakan versi lebih sederhana dari SGML.
Justru inti utama dari HTML (dan HTTP) adalah di kesederhanaannya. Itulah
salah satu alasan mengapa Web menjadi sangat populer. Sebelumnya sudah ada
berbagai inisiatif untuk membangun sistem {\em hypertext}, tetapi gagal.

HTTP merupakan sebuah protokol untuk mendistribusikan data.  Protokol HTTP
didesain dengan sifat {\em connectionless} - lagi-lagi - agar
sederhana~\cite{RFC2068}. Protokol ini kemudian berkembang sesuai dengan
kebutuhan dan kemajuan jaman. Sebagai contoh, nantinya akan ada protokol HTTPS
dan seterusnya. Bahasan mengenai kemananan akan banyak terkait dengan protokol
HTTP (dan turunannya) ini.

Salah satu alasan mengapa Web menjadi populer saat ini adalah karena dia
memudahkan dalam operasional karena sifatnya yang
tersentralisasi~\footnote{Dalam perkembangan teknologi selalu ada ayunan
(swing) dari tersentralisasi dan terdistribusi. Pada jaman {\em mainframe},
sistem komputer tersentralisasi. Mainframe merupakan pusat dan pengguna
menggunakan (green) terminal untuk mengakses mainframe secara bergantian.
Teknologi kemudian berubah menjadi terdistribusi dengan adanya sistem {\em
client server}. Di kantor-kantor cabang ada server lokal yang terhubung ke
kantor pusat secara terjadwal. Teknologi web kembali menjadi tersentralisasi.}.
Salah satu masalah dalam sistem {\em client server} adalah ketika terjadi
update software. Sangat repot untuk melakukan update di sisi {\em client}.
(Apalagi bagi instansi yang mempunyai banyak kantor cabang.) Sementara itu
dengan menggunakan web, update hanya perlu dilakukan di server web saja.

Dalam sistem web, klien hanya perlu memasang (web) browser. Web browser ini
juga tadinya hanya dapat memperagakan web secara sederhana. Perkembangan
teknologi kemudian memperkenankan penambahan fitur di browser dengan
menggunakan konsep {\em plugin}. {\em Interactivity} kemudian diberikan dengan
memasang (Adobe) Flash plugin. Namun ada kekhawatiran bahwa plugin-plugin ini
dapat menyebabkan sumber masalah keamanan.

\section{Topologi dan Asumsi}
Secara umum, sistem Web terdiri dari tiga hal:
\begin{itemize}
   \item web server (Apache, IIS, Nginx, Tomcat);
   \item web browser (Firefox, Chrome, Internet Explorerd (IE), Safari, 
      Edge, Lynx, Links, Opera, Vivaldi, Brave, Galeon, Kfm,
      dan program seperti wget, curl);
   \item jaringan (yang menghubungankan web server dan web browser).
\end{itemize}

Ada beberapa asumsi awal yang kita gunakan agar sistem Web berjalan
dengan aman. Asumsi-asumsi inilah yang dilanggar ketika terjadi masalah
keamanan\footnote{Jika diperhatikan dengan seksama, sebenarnya asumsi ini
terkait dengan aspek {\em confidentiality}, {\em integrity}, dan {\em
availability}}.

Asumsi terkait dengan web server adalah sebagai berikut.

\begin{enumerate}
\item Web (server) dimiliki oleh organisasi atau institusi yang benar dan
terkait dengan nama domain yang digunakan. Sebagai contoh, web dari BNI
dimiliki oleh Bank BNI.
\item Dokumen yang disajikan (diberikan) oleh web tidak mengandung malware
(virus, trojan horse, dan seterusnya).
\item Data catatan (logging) di server web tidak digunakan untuk keperluan yang
tidak semestinya. Sebagai contoh, data log tersebut tidak boleh
diperjual-belikan atau dipublikasikan secara umum.
\end{enumerate}

Asumsi terkait dengan klien yang menggunakan web browser adalah sebagai
berikut.
\begin{enumerate}
   \item Identitas pengguna adalah benar dan pengguna memiliki itikad baik
      dalam mengakses web server.
   \item Pengguna hanya mengambil data dari direktori yang diperkenankan.
   \item Pengguna memiliki itikad baik, tidak berniat merusak server web.
\end{enumerate}

Asumsi terkait dengan jaringan yang menghubungan web server dan web browser
adalah sebagai berikut.
\begin{enumerate}
   \item Jaringan (dan komputer) bebas dari penyadapan, pengubahan, serangan
      DoS ({\em Denial of Service}), 
      dan serangan-serangan lainnya. (Masalah terkait dengan jaringan akan menjadi
      masalah {\em network security}, bukan masalah keamanan web.)
\end{enumerate}

Asumsi-asumsi di atas akan dilanggar oleh penyerang, yang mengakibatkan masalah
keamanan sistem web.


\section{Keamanan Server Web}
Server web menyediakan informasi secara statis dan dinamis. Halaman statis
diperoleh dengan perintah {\tt GET}. Halaman dinamis diperoleh dengan berbagai
cara;
\begin{itemize}
   \item CGI (Common Gateway Interface) dengan menggunakan berbagai bahasa
      pemrograman seperti shell script, PHP, Perl, Python, dan seterusnya;
   \item Server Side Include (SSI);
   \item Active Server Page (ASP);
   \item Servlets.
\end{itemize}
