\chapter{Keamanan Sistem Web}
World Wide Web (WWW, atau sering disingkat menjadi Web saja) merupakan aplikasi
yang dominan pada saat ini. Kepopulerannya setara dengan aplikasi email. Boleh
jadi kalau dilihat dari jumlah data yang dikirimkan melalui web atau melalui
email, maka jumlah data yang dikirim melalui web lebih besar.

Bahasan mengenai masalah keamanan sistem web sesungguhnya dapat menjadi sebuah
buku yang tersendiri. Bab ini akan membahas secara singkat masalah-masalah
tersebut.

\section{Sejarah Web}
WWW pada awalnya dikembangkan oleh Tim Berners-Lee ketika dia bekerja di CERN,
sebuah tempat penelitian terkait dengan Fisika di Swiss di akhir tahun 1980-an
(sekitar tahun 1989).
CERN merupakan tempat berkumpulnya peneliti-peneliti yang berasal dari berbagai
tempat di seluruh dunia. Komputer dan software yang digunakan para peneliti
tersebut tentunya berbeda-beda. Demikian pula dokumen yang dihasilkannya
diletakkan pada komputer yang tersebar.

Pada saat itu protokol untuk distribusi informasi yang paling banyak digunakan
adalah FTP dan Gopher. Keduanya cukup baik untuk mengelola atau
mendistribusikan data (informasi) dalam sebuah direktori, misalnya. Namun orang
masih harus mencari dimana letak direktori tersebut (di komputer mana dan dalam
direktori mana). Proses ini membuat distribusi informasi tidak terlalu nyaman.

WWW terdiri dari dua komponen utama, yaitu standar format dokumen yang
disebut HTML (HyperText Markup Language) dan protokol HTTP (HyperText Transfer
Protocol). Keduanya dikembangkan oleh Tim Berners-Lee~\cite{webbook}.

HTML merupakan sebuah standar yang dikembangkan dari SGML, Standard Generalized
Markup Language. Sebenarnya HTML merupakan versi lebih sederhana dari SGML.
Justru inti utama dari HTML (dan HTTP) adalah di kesederhanaannya. Itulah
salah satu alasan mengapa Web menjadi sangat populer. Sebelumnya sudah ada
berbagai inisiatif untuk membangun sistem {\em hypertext}, tetapi gagal.

HTTP merupakan sebuah protokol untuk mendistribusikan data.  Protokol HTTP
didesain dengan sifat {\em connectionless} - lagi-lagi - agar
sederhana~\cite{RFC2068}. Protokol ini kemudian berkembang sesuai dengan
kebutuhan dan kemajuan jaman. Sebagai contoh, nantinya akan ada protokol HTTPS
dan seterusnya. Bahasan mengenai kemananan akan banyak terkait dengan protokol
HTTP (dan turunannya) ini.

\section{Topologi dan Asumsi}
Secara umum, sistem Web terdiri dari tiga hal, yaitu web server, web browser
(sebagai klien), dan jaringan (yang menghubungkan antara web server dan web 
browser). Ada beberapa asumsi awal yang kita gunakan agar sistem Web berjalan
dengan aman. Asumsi-asumsi inilah yang dilanggar ketika terjadi masalah
keamanan\footnote{Jika diperhatikan dengan seksama, sebenarnya asumsi ini
terkait dengan aspek {\em confidentiality}, {\em integrity}, dan {\em
availability}}.

Asumsi terkait dengan web server adalah sebagai berikut.

\begin{enumerate}
\item Web (server) dimiliki oleh organisasi atau institusi yang benar dan
terkait dengan nama domain yang digunakan. Sebagai contoh, web dari BNI
dimiliki oleh Bank BNI.
\item Dokumen yang disajikan (diberikan) oleh web tidak mengandung malware
(virus, trojan horse, dan seterusnya).
\item Data catatan (logging) di server web tidak digunakan untuk keperluan yang
tidak semestinya. Sebagai contoh, data log tersebut tidak boleh
diperjual-belikan atau dipublikasikan secara umum.
\end{enumerate}

Asumsi terkait dengan klien yang menggunakan web browser adalah sebagai
berikut.
\begin{enumerate}
   \item Identitas pengguna adalah benar dan pengguna memiliki itikad baik
      dalam mengakses web server.
   \item Pengguna hanya mengambil data dari direktori yang diperkenankan.
\end{enumerate}

Asumsi terkait dengan jaringan yang menghubungan web server dan web browser
adalah sebagai berikut.
\begin{enumerate}
   \item Jaringan bebas dari penyadapan, pengubahan, serangan DoS, dan
      serangan-serangan lainnya. (Masalah terkait dengan jaringan akan menjadi
      masalah {\em network security}, bukan masalah keamanan web.)
\end{enumerate}

Asumsi-asumsi di atas akan dilanggar oleh penyerang, yang mengakibatkan masalah
keamanan sistem web.
