\chapter{Pengantar}

Buku ini muncul karena kebutuhan buku teks untuk kuliah
keamanan informasi ({\em information security}).
Jenis buku seperti ini agak langka.
Bahkan dahulu ilmu yang terkait dengan keamanan - misalnya kriptografi -
dianggap tidak boleh diajarkan sehingga referensi untuk hal itu
sangat langka.
Buku yang pertama kali terbit mengenai kriptografi adalah
``Codebreakers'' karangan David Kahn~\cite{davidkahn},
yang diterbitkan tahun 1969.

Sebelumnya saya pernah membuat buku yang sejenis, tetapi kode sumber
dari buku tersebut sudah hilang.
Maklum, saya membuatnya di tahun 1990-an dengan menggunakan program
FrameMaker, yang sudah tidak saya miliki lagi.
Sekarang saya buat dari awal dengan menggunakan \LaTeX \ agar
lebih bisa bebas.

Bagi Anda yang mengajarkan kuliah {\em security} dan ingin menggunakan
buku ini sebagai buku teks, silahkan digunakan.
Bagi para mahasiswa dan peneliti yang membutuhkan referensi untuk
makalah Anda, semoga buku ini dapat membantu.

Selamat menikmati versi 0.1 dari buku ini.
\vspace{5 mm}

Bandung, 2017


Budi Rahardjo, peneliti\\
twitter: @rahard\\
blog: http://rahard.wordpress.com

\vspace{5 mm}
Penulisan referensi:\\
Budi Rahardjo, {\em ``Keamanan Informasi''}, PT Insan Infonesia, 2017.
