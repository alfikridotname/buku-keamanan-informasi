\chapter{Pengantar}

Buku ini muncul karena kebutuhan buku teks untuk kuliah
keamanan informasi ({\em information security}).
Jenis buku seperti ini agak langka.
Bahkan dahulu ilmu yang terkait dengan keamanan - misalnya kriptografi -
dianggap tidak boleh diajarkan sehingga referensi untuk hal itu
sangat langka.
Buku yang pertama kali terbit mengenai kriptografi adalah
``Codebreakers'' karangan David Kahn~\cite{davidkahn},
yang diterbitkan tahun 1969.
Sejak saat ini, ilmu tentang keamanan (security) mulai terbuka untuk umum.

Buku teks berbeda dengan buku {\em how to} yang banyak beredar
di toko buku. Buku tersebut biasanya hanya menjelaskan bagaimana
menggunakan sebuah program tertentu, atau melakukan hal tertentu.
Sementara itu buku teks digunakan untuk memberikan landasan teori
sehingga pemahaman tidak bergantung kepada {\em tools} tertentu saja.
Meskipun demikian, penggunaan {\em tools} sebagai contoh akan
juga disampaikan dalam buku ini.
Semoga dengan demikian, buku ini dapat bertahan lebih lama.
(Meskipun saya agak ragu setelah melihat pesatnya perkembangan
teknologi informasi.)

Urutan pembahasan juga membuat saya merenung cukup panjang.
Ada beberapa hal yang disinggung di depan, tetapi pembahasan teorinya
di belakang. Sementara itu kalau teorinya diletakkan di depan,
maka siswa akan bosan karena terlalu banyak teori.
Seharusnya memang buku ini dipaketkan dengan materi presentasi
(slide) yang saya gunakan untuk mengajar. 
Yang itu belum saya benahi. Masih menunggu waktu.

Sebelumnya saya pernah membuat buku yang sejenis, tetapi kode sumber
dari buku tersebut sudah hilang.
Maklum, saya membuatnya di tahun 1990-an dengan menggunakan program
FrameMaker, yang sudah tidak saya miliki lagi.
Sekarang saya buat dari awal dengan menggunakan \LaTeX \ agar
lebih bisa bebas.

Ada yang mengatakan bahwa {\em security} itu seperti pisau yang bermata dua.
Dia dapat digunakan untuk kebaikan atau kejahatan, tergantung kepada
penggunanya. Saya berharap agar ilmu yang diperoleh dari membaca buku ini dapat
digunakan untuk kebaikan, bukan kejahatan. Saat ini masih dibutuhkan banyak
tenaga kerja yang menguasai security (security professionals). Sayang sekali
kalau lowongan ini tidak dapat dipenuhi dan malah banyak yang memilih untuk
menjadi perusak.

Bagi Anda yang mengajarkan kuliah {\em security} dan ingin menggunakan
buku ini sebagai buku teks, silahkan digunakan.
Bagi para mahasiswa dan peneliti yang membutuhkan referensi untuk
makalah Anda, semoga buku ini dapat membantu.
Lebih baik lagi apabila Anda dapat menemukan guru yang dapat membantu Anda
dalam memahami isi buku ini.
Selain buku ini, saya juga menulis buku lain yang dapat diunduh juga:
``Keamanan Perangkat Lunak''~\cite{BRsecuresoftware}.
Yang ini saya gunakan untuk kuliah saya yang lainnya.

Selain prinsip-prinsip keamanan, ada beberapa {\em tools} yang juga diuraikan
di buku ini karena mereka sering digunakan, seperti misalnya program tcpdump.
Tentu saja Anda dapat membaca berbagai tutorial tentang penggunaan tools
tersebut. Keberadaan tulisan tersebut agar buku ini komplit.

Dikarenakan buku ini masih dalam pengembangan, maka ada banyak bagian yang
masih kosong atau meloncat. Mohon dimaafkan. Dalam penulisan selanjutnya,
bagian-bagian tersebut akan diisi dan dilengkapi. Mohon masukkan jika hal ini
terjadi.

Selamat menikmati versi 0.4 dari buku ini. Semoga bermanfaat.
\vspace{5 mm}

Bandung, 2017


Budi Rahardjo, peneliti\\
twitter: @rahard\\
blog: http://rahard.wordpress.com\\
web: http://budi.rahardjo.id

\vspace{5 mm}
Penulisan referensi:\\
Budi Rahardjo, {\em ``Keamanan Informasi''}, PT Insan Infonesia, 2017.

\doclicenseThis
