\chapter{Pendahuluan}
Selalu ada aspek negatif dari sebuah pemanfaatan teknologi.
Teknologi informasi tidak lepas dari masalah ini.
Ada banyak manfaat dari teknologi informasi.
Sayangnya salah satu aspek negatifnya adalah masalah keamanan
({\em security}).

Banyak tulisan dan buku yang mengajarkan cara merusak sebuah
sistem informasi. Sementara itu buku yang mengajarkan cara
pengamanannya agak minim. Demikian pula, ilmu untuk mengamankan
sistem berbasis teknologi informasi juga harus lebih banyak
diajarkan.

%%%
\section{Keamanan Informasi}
Ketika kita berbicara tentang {\em security}, yang muncul dalam
benak kebanyakan orang adalah {\em network security}, keamanan jaringan.
Padahal sesungguhnya yang ingin kita amankan adalah \textbf{informasi}.
Bahwa informasi tersebut dikirimkan melalui jaringan adalah benar,
tetapi tetap yang ingin kita amankan adalah informasinya.
Nanti akan kita bahas lebih lanjut mengapa demikian.
Maka judul dari buku ini adalah ``Keamanan Informasi''.

%%%
\section{Beberapa Kasus}
Untuk menunjukkan betapa banyaknya masalah keamanan informasi,
berikut ini ada beberapa contoh kasus-kasus.
Contoh ini bukanlah daftar yang komplit, melainkan hanya sampel
dari kondisi yang ada. Bahkan, kemungkinan kondisi yang ada
lebih parah daripada contoh-contoh ini.
