\chapter{Penanganan Insiden Teknologi Informasi}
Sebaik-baiknya kita menjaga keamanan, masih tetap ada kemungkinan terjadi
insiden. Ini mirip dengan kondisi kesehatan kita. Meskipun kita sudah berusaha
sebaik mungkin untuk menjaga kesehatan kita, tetapi masih ada kemungkinan kita
jatuh sakit. Bagaimana kesiapan kita dalam menghadapi atau menangani insiden
terkait dengan teknologi informasi? 

Penanganan insiden dapat dilakukan secara {\em ad hoc}, tanpa direncanakan
apabila kejadian insiden tersebut jarang terjadi. Misalnya insiden hanya
terjadi dua tahun sekali. Pasalnya insiden teknologi informasi makin sering
terjadi dan pengaruhnya terhadap perusahaan atau instansi menjadi signifikan.
Hal ini disebabkan pemanfaatan teknologi informasi makin merasuk dalam kegiatan
sehari-hari bisnis dan aktivitas kita lainnya.

Insiden dapat terjadi karena tidak sengaja (misal karena bencana
alam) atau sengaja (misal adanya orang yang jahat).
Insiden ini sering terjadi pada waktu yang ``kurang pas''.
Sebagai contohnya insiden terjadi ketika hari libur sehingga sebagian
besar anggota tim (apalagi personel yang utama) sedang tidak ada
di tempat.

Beberapa contoh insiden terkait dengan teknologi informasi,
antara lain:
\begin{itemize}
\item wabah virus (malware, termasuk ransomware);
\item email spam;
\item serangan Denial of Service (DoS);
\item kebocoran informasi;
\item penyadapan data;
\item dan penyusupan oleh orang yang tidak berhak dan beritikad jahat.
\end{itemize}
Selain contoh-contoh di  atas tentunya masih banyak insiden lainnya.

Sedemikian pentingnya kemampuan penanganan insiden sehingga ISO 27002,
{\em Information technology - Security techniques - Code of practice
for information security management},
menempatkan kemampuan penanganan insiden menjadi salah satu poin utamanya.

\begin{enumerate}
\item Security Policy
\item Organization of Information Security
\item Asset Management
\item Human Resource Security
\item Physical Security
\item Communications and Operation Management
\item Access Control
\item Information Systems Acquisition, Development, Maintenance
\item Information Security Incident Management
\item Business Continuity
\item Compliance
\end{enumerate}

\section{Definisi Insiden}
Agar tidak terjadi kerancuan, maka perlu didefinisikan istilah ``insiden''.
Ada berbagai definisi dari insiden. Berikut ini adalah Beberapa
contoh dari definisi tersebut.

David Theunissen dalam ``Corporate Incident Handling Guidelines'':
\begin{quote}
... the act of violating or threatening to violate an explicit or 
implied security policy.
\end{quote}
Perhatikan definisi tersebut mengatakan bahwa insiden adalah segala
kegiatan yang melanggar atau mengancam pelanggaran terhadap
kebijakan keamanan. Dengan kata lain, harus ada kebijakan keamanan
dahulu sebelum kita dapat mendefinisikan insiden.
Tanpa ada kebijakan keamanan maka tidak ada istilah insiden di
dalam instansi tersebut.


Sementara itu Kevin Mandia dan Chris Prosise dalam bukunya
``Incident Response'' mendefinisikan insiden sebagai berikut:
\begin{quote}
   Incidents are events that interrupt normal operating procedure
   and precipitate some level of crisis.
\end{quote}
Mirip dengan definisi sebelumnya, insiden dikaitkan dengan adanya
prosedur normal. Untuk itu perlu didefinisikan dahulu apa yang
disebut ``normal''. Biasanya ini dikaitkan dengan keberadaan
standar atau prosedur di instansi tersebut.


Tujuan dari penanganan insiden adalah:
\begin{itemize}
   \item memastikan bahwa insiden terjadi atau tidak terjadi;
   \item melakukan informasi yang akurat;
   \item melakukan pengambilan dan penanganan bukti-bukti
      (menjaga {\em chain of custody});
   \item menjaga agar kegiatan berada dalam kerangka hukum
      (misal adalah masalah privasi dan hukum);
   \item meminimalkan gangguan terhadap operasi bisnis dan
      sistem teknologi informasi;
   \item membuat laporan yang akurat beserta rekomendasinya.
\end{itemize}


\section{Latihan Penanganan Keamanan}
Salah satu kegiatan terkait dengan penanganan insiden adalah melatih kemampuan
penanganan insiden. Biasanya kegiatan ini disebut sebagai {\em exercise} atau
{\em drill}. Sebetulnya latihan ini tidak berdiri sendiri, melainkan terkait
dengan kegiatan-kegiatan lain seperti pelatihan dan pengujian.
Salah satu dokumen yang dapat digunakan sebagai acuan untuk melakukan pelatihan
adalah NIST SP800-84~\cite{sp800-84}.
