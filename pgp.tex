\chapter{PGP / Gnu Privacy Guard}
Pretty Good Privacy (PGP) pada awalnya adalah aplikasi yang dapat digunakan
pengguna untuk menggunakan kriptografi di berbagai aplikasi dengan lebih mudah.
Pengembangan selanjutnya PGP menjadi bagian dari {\em public key
infrastructure}.

\section{Sejarah}
[... more to be written ...]

Gnu Privacy Guard (GPG) merupakan implementasi dari PGP yang bersifat terbuka.
(Catatan: Singkatan dari GPG ini merupakan guyonan terhadap PGP.) Bab ini akan
membahas lebih banyak tentang GGP, meskipun konsep yang sama dapat juga
diterapkan pada PGP jika Anda menggunakan produk PGP yang komersial.

Dalam buku ini, kita akan menggunakan GPG versi {\em command line interface},
yaitu dengan mengetikkan perintah ``gpg'' di program terminal atau CMD.exe.
Ada banyak program {\em GUI} dari GPG ini. Silahkan gunakan manual terkait
dengan program-program tersebut. Prinsipnya masih tetap sama.

\section{Menggunakan Gnu Privacy Guard, gpg}
Awal dari penggunakan GPG adalah membuat pasangan kunci publik dan privat. Hal
ini dapat dilakukan dengan menggunakan perintah berikut.

\begin{verbatim}
gpg --key-gen
\end{verbatim}

Perintah di atas akan menanyakan beberapa hal, seperti jenis algoritma yang
digunakan (pilih RSA dan RSA), panjang kuncinya (pilih 2048), dan alamat email
yang akan digunakan untuk kunci tersebut. Dalam contoh buku ini saya akan
menggunakan alamat email ``rahard2017@gmail.com''. Gunakan alamat email Anda
sebagai penggantinya/

Setelah proses {\em key generation} selesai, kunci publik dapat diekspor dengan
menggunakan perintah  berikut. Gantikan ``rahard2017@gmail.com'' dengan alamat
email Anda.

\begin{verbatim}
gpg --export --armor rahard2017@gmail.com > kunci-public.asc
\end{verbatim}

Akan dihasilkan berkas ``kunci-public.asc''. Tanpa perintah {\em redirect
output} yang menggunakan ``>'' itu, hasilnya hanya akan ditampilkan di layar
saja. Tentu saja Anda dapat menggunakan nama berkas lainnya.

Untuk melihat informasi mengenai kunci Anda, dapat digunakan perintah berikut:

\begin{verbatim}
gpg --fingerprint rahard2017

pub   2048R/EB6CEB46 2017-02-14 [expires: 2017-03-07]
      Key fingerprint = 810B 2149 3366 E699 F95E  9E49 07E1 BDC5 EB6C EB46
uid       [ultimate] Budi Rahardjo <rahard2017@gmail.com>
sub   2048R/C2E83C60 2017-02-14 [expires: 2017-03-07]
\end{verbatim}

Perhatikan bahwa kunci publik yang ini memiliki ID ``EB6CEB46''. ID ini dapat
digunakan untuk bertukar kunci, atau untuk melakukan proses-proses lainnya.
Sebagai contoh, kita dapat mengirimkan kunci ini ke server agar dapat dilihat
atau dicari oleh orang lain. Keyserver yang akan kita gunakan dalam contoh ini
adalah pgp.mit.edu.

\begin{verbatim}
gpg --keyserver pgp.mit.edu --send-keys EB6CEB46
\end{verbatim}
