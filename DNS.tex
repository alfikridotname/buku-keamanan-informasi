\chapter{Domain Name System (DNS)}
Salah satu komponen utama dari internet adalah penamaan domain, Domain Name
System (DNS). Penelusuran DNS juga merupakan salah satu langkah awal dari
evaluasi keamanan atau penyerangan.

\section{Sejarah DNS}
Komputer bekerja dengan angka, atau lebih tepatnya dengan menggunakan bilangan
biner, ON dan OFF. Salah satu cara memberikan identitas komputer adalah dengan
menggunakan nomor IP. Contoh nomor IP antara lain dapat dilihat di bawah ini.
(Tahukah Anda itu nomor IP dari layanan apa?)

\begin{verbatim}
172.217.27.3
31.13.78.35
104.244.42.193
\end{verbatim}

Di sisi lain, manusia lebih mudah menghapal nama dibandingkan dengan angka.
Oleh sebab itu harus ada sebuah sistem yang melakukan konversi dari nama ke
angka dan sebaliknya. Pada awalnya, hal ini dapat dilakukan dengan menggunakan
sebuah tabel. Di sistem UNIX, tabel itu ada di berkas {\em /etc/hosts}

\begin{verbatim}
167.205.24.34   paume
192.168.100.1   br
\end{verbatim}

Pada awalnya, tabel ini jumlahnya tidak banyak sehingga dapat dikelola secara
manual. Inilah yang dilakukan oleh Jon Postel dan kawan-kawannya di Amerika
Serikat. Dikarenakan pengelolaannya menjadi semakin berat, maka dibentuklah
sebuah organisasi yang diberi nama {\em Internet Assigned Numbers Authority}
(IANA)\footnote{IANA juga yang mengatur pembagian atau distribusi nomor IP.}.
Tabel-tabel ini diperbaharui secara berkala dan dapat diunduh melalui FTP.

Jumlah komputer dan organisasi yang terhubung ke internet semakin bertambah
sehingga pengelolaan nama yang tersentralisai ini menjadi makin rumit. Ditambah
lagi ada perebutan nama yang populer (seperti server, mail, dan seterusnya).
Nama yang sama tidak diperbolehkan dalam sistem yang menggunakan tabel
tersebut. Akhirnya disepakati harus ada sebuah sistem yang terdistribusi
pengelolaannya. Maka lahirlah DNS.
