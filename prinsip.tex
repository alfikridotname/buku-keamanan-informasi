\chapter{Prinsip-prinsip Keamanan Informasi}

Ada beberapa prinsip utama dalam keamanan informasi.
Bab ini akan membahas prinsip-prinsip tersebut.


\section{Aspek Keamanan}
Ketika kita berbicara tentang keamanan informasi, maka yang kita
bicarakan adalah tiga hal;
{\em confidentiality}, {\em integrity}, dan {\em availability}.
Ketiganya sering disebut dengan istilah \textbf{CIA},
yang merupakan gabungan huruf depan dari kata-kata tersebut.
Selain ketiga hal tersebut, masih ada aspek keamanan lainnya.

\subsection{Confidentiality}
{\em Confidentiality} atau kerahasiaan adalah aspek yang biasa dipahami
tentang keamanan.
Aspek confidentiality menyatakan bahwa data hanya dapat diakses
atau dilihat oleh orang yang berhak.
Biasanya aspek ini yang paling mudah dipahami oleh orang.
Jika terkait dengan data pribadi, aspek ini juga dikenal dengan
istilah {\em Privacy}.

Serangan terhadap aspek confidentiality dapat berupa
penyadapan data (melalui jaringan),
memasang {\em keylogger} untuk menyadap apa-apa yang diketikkan
di keyboard,
dan pencurian fisik mesin / disk yang digunakan untuk menyimpan data.

Perlindungan terhadap aspek {\em confidentiality} dapat dilakukan
dengan menggunakan kriptografi, 
dan membatasi akses (segmentasi jaringan)


\subsection{Integrity}
Aspek {\em integrity} mengatakan bahwa data tidak boleh berubah
tanpa ijin dari yang berhak.
Sebagai contoh, jika kita memiliki sebuah pesan atau data 
transaksi di bawah ini (transfer dari rekening 12345 ke rekening 6789 
dengan nilai transaksi teretentu), 
maka data transaksi tersebut tidak dapat diubah seenaknya.

\begin{verbatim}
TRANSFER 12345 KE 6789 100000
\end{verbatim}

Serangan terhadap aspek {\em integrity} dapat dilakukan oleh
{\em man-in-the-middle}, yaitu menangkap data di tengah jalan
kemudian mengubahnya dan meneruskannya ke tujuan.
Data yang sampai di tujuan (misal aplikasi di web server) tidak tahu
bahwa data sudah diubah di tengah jalan.

Perlindungan untuk aspek {\em integrity} dapat dilakukan dengan
menggunakan {\em message authentication code}.


\subsection{Availability}
Ketergantungan kepada sistem yang berbasis teknologi informasi
menyebabkan sistem (beserta datanya) harus dapat diakses ketika dibutuhkan.
Jika sistem tidak tersedia, {\em not available}, maka dapat terjadi
masalah yang menimbulkan kerugian finansial atau bahkan nyawa.
Itulah sebabnya aspek {\em availability} menjadi bagian dari keamanan.

Serangan terhadap aspek {\em availability} dilakukan dengan tujuan
untuk meniadakan layanan atau membuat layanan menjadi sangat lambat
sehingga sama dengan tidak berfungsi.
Serangannya disebut {\em Denial of Service} (DOS).

Perlindungan terhadap aspek {\em availability} dapat dilakukan
dengan menyediakan redundansi.
Sebagai contoh, jaringan komputer dapat menggunakan layanan dari
dua penyedia jasa yang berbeda.
Jika salah satu penyedia jasa jaringan mendapat serangan (atau rusak),
maka masih ada satu jalur lagi yang dapat digunakan.
