\chapter{Kriptografi}
Ada dua cara untuk mengamankan data, yaitu menyembunyikan data atau menyandikan
data. Cara pertama menggunakan steganografi, sementara cara kedua menggunakan
kriptografi. Bab ini akan membahas lebih banyak tentang kriptografi, meskipun
steganografi akan disinggung secara singkat.

Ilmu ini pada awalnya dianggap terlarang untuk diajarkan sehingga tidak ada
bahan bacaan untuk mempelajarinya. Setelah David Kahn membuat bukunya di tahun
1969, maka ilmu pengamanan data ini menjadi lebih terbuka untuk dipelajari.
Saat ini sudah sangat banyak buku yang membahas mengenai hal ini, mulai dari
yang umum~\cite{levycrypto} (tidak teknis) sampai ke yang teknis.


\section{Steganografi}
Steganografi ({\em steganography}) adalah ilmu untuk menyembunyikan pesan
sehingga tidak terlihat dengan mudah. Mekanisme penyembunyian ({\em hide}, {\em
concealment}) dilakukan dengan menggunakan media lain. Sebagai contoh, kita
dapat menyembunyikan pesan dalam gambar ({\em image}, foto), audio, atau video.
Dalam sejarahnya, penyembunyian pesan dapat dilakukan dengan menggunakan meja
yang dilapisi lilin (jaman perang antara Yunani dan Persia).

[Masih harus ditulis.]
