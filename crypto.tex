\chapter{Kriptografi}
Ada dua cara untuk mengamankan data, yaitu menyembunyikan data atau menyandikan
data. Cara pertama menggunakan steganografi, sementara cara kedua menggunakan
kriptografi. Bab ini akan membahas lebih banyak tentang kriptografi, meskipun
steganografi akan disinggung secara singkat.

Ilmu ini pada awalnya dianggap terlarang untuk diajarkan sehingga tidak ada
bahan bacaan untuk mempelajarinya. Setelah David Kahn membuat bukunya di tahun
1969, maka ilmu pengamanan data ini menjadi lebih terbuka untuk dipelajari.
Saat ini sudah sangat banyak buku yang membahas mengenai hal ini, mulai dari
yang umum~\cite{levycrypto} (tidak teknis) sampai ke yang teknis.


\section{Steganografi}
Steganografi ({\em steganography}) adalah ilmu untuk menyembunyikan pesan
sehingga tidak terlihat dengan mudah. Mekanisme penyembunyian ({\em hide}, {\em
concealment}) dilakukan dengan menggunakan media lain. Sebagai contoh, kita
dapat menyembunyikan pesan dalam gambar ({\em image}, foto), audio, atau video.
Dalam sejarahnya, penyembunyian pesan dapat dilakukan dengan menggunakan meja
yang dilapisi lilin (jaman perang antara Yunani dan Persia).

Saat ini steganografi digunakan sebagai bagian dari {\em Digital Rights
Management} (DRM), misalnya dengan menyisipkan informasi mengenai HaKI dari
produk digital (musik, ebook, foto, dan sejenisnya).

[Masih harus ditulis.]



\section{Kriptografi}
Berbeda dengan steganografi, kriptografi tidak menyembunyikan pesan tetapi
mengubah pesan sehingga sulit diperoleh pesan aslinya. Pesan diubah dengan cara
{\bf transposisi} (mengubah letak dari huruf) dan {\bf substitusi} (mengganti
huruf/kata dengan huruf/kata lainnya). Pesan yang sudah diubah terlihat seperti
sampah, tetapi tetap terlihat oleh penyerang atau orang yang tidak berhak.

Proses transposisi dapat dilakukan dengan berbagai cara. Sebagai contoh, kita
dapat menulis pesan menjadi dua baris secara bergantian. Huruf pertama
diletakkan di baris pertama, huruf kedua di baris kedua, huruf ketiga di baris
pertama lagi, huruf keempat di baris kedua lagi, dan seterusnya. Mari kita
ambil contoh kalimat ``selamat datang di kota bandung''. Kalimat ini akan kita
tuliskan secara bergantian. (Dalam contoh ini, spasi kita buang saja.)

\begin{mdframed}
\begin{verbatim}
slmtaagioaadn
eaadtndktbnug
\end{verbatim}
\end{mdframed}

Kalimat yang akan dikirimkan menjadi ``slmtaagioaadneaadtndktbnug''. Kalimat
ini yang kita kirimkan. Di sisi penerima akan dilakukan proses sebaliknya
sehingga didapat kalimat aslinya.


Secara umum ada tiga komponen utama dari kriptografi, yaitu {\em plain text},
{\em ciphered text}, dan algoritma serta kunci yang digunakan.
